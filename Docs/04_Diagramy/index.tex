%%%%%%%%%%%%%%%%%%%%%%%%%%%%%%%%%%%%%%%%%
% Academic Title Page
% LaTeX Template
% Version 2.0 (17/7/17)
%
% This template was downloaded from:
% http://www.LaTeXTemplates.com
%
% Original author:
% WikiBooks (LaTeX - Title Creation) with modifications by:
% Vel (vel@latextemplates.com)
%
% License:
% CC BY-NC-SA 3.0 (http://creativecommons.org/licenses/by-nc-sa/3.0/)
% 
% Instructions for using this template:
% This title page is capable of being compiled as is. This is not useful for 
% including it in another document. To do this, you have two options: 
%
% 1) Copy/paste everything between \begin{document} and \end{document} 
% starting at \begin{titlepage} and paste this into another LaTeX file where you 
% want your title page.
% OR
% 2) Remove everything outside the \begin{titlepage} and \end{titlepage}, rename
% this file and move it to the same directory as the LaTeX file you wish to add it to. 
% Then add \input{./<new filename>.tex} to your LaTeX file where you want your
% title page.
%
%%%%%%%%%%%%%%%%%%%%%%%%%%%%%%%%%%%%%%%%%

%----------------------------------------------------------------------------------------
%	PACKAGES AND OTHER DOCUMENT CONFIGURATIONS
%----------------------------------------------------------------------------------------
\documentclass[12pt,a4paper,titlepage,hidelinks]{article}

\usepackage[utf8]{inputenc}
\usepackage[polish]{babel}
\usepackage[T1]{fontenc}
\usepackage{subfiles}
\usepackage{float}
\usepackage{hyperref}
\usepackage{color}
\usepackage{amsmath}
\usepackage{hyperref}
\usepackage{graphicx}

\usepackage{mathpazo} % Use the Palatino font by default


\newcommand\myemptypage{
	\null
	\thispagestyle{empty}
	% \addtocounter{page}{-1}
	\newpage
}

\newcommand\pageCounterMinusOne{
	\addtocounter{page}{-1}
}

\title{\Huge{Multiplatformowe Aplikacje Mobilne}}
\author{Jakub Barylak\\Piotr Marcol}
\date{\today}


\begin{document}

\selectlanguage{polish}


\begin{titlepage} % Suppresses displaying the page number on the title page and the subsequent page counts as page 1
    \newcommand{\HRule}{\rule{\linewidth}{0.5mm}} % Defines a new command for horizontal lines, change thickness here

    \center % Centre everything on the page

    %------------------------------------------------
    %	Headings
    %------------------------------------------------

    \textsc{\LARGE Politechnika Śląska}\\[0.5cm] % Main heading such as the name of your university/college

    \textsc{\Large Wydział Automatyki, Elektroniki i Informatyki}\\[1cm] % Major heading such as course name

    \textsc{\large Multiplatformowe Aplikacje Mobilne}\\[0.5cm] % Minor heading such as course title

    %------------------------------------------------
    %	Title
    %------------------------------------------------

    \HRule\\[0.4cm]

    { {\Huge\bfseries Opis Projektu}\\[0.2cm] {\Large Aplikacja do wewnętrznego udostępniania zdjęć w trakcie wydarzeń}}\\[0.2cm] % Title of your document

    \HRule\\[1.5cm]

    %------------------------------------------------
    %	Author(s)
    %------------------------------------------------

    \begin{minipage}{0.5\textwidth}
        \begin{flushleft}
            % \large
            \textsc{\bfseries
                Jakub Barylak
                \\Piotr Marcol}\\
            Informatyka SSI
            \\sem. 6, ISMIP\\
            Rok akademicki 2023/2024
        \end{flushleft}
    \end{minipage}
    % ~
    \begin{minipage}{0.4\textwidth}
        \begin{flushright}
            % \large
            \textit{Prowadzący}\\
            dr inż. Michał Sawicki
        \end{flushright}
    \end{minipage}



    %------------------------------------------------
    %	Logo
    %------------------------------------------------

    % \vfill\vfill
    % \includegraphics[width=0.5\textwidth]{polslLogo.jpeg}\\[1cm] % Include a department/university logo - this will require the graphicx package

    %----------------------------------------------------------------------------------------

    %------------------------------------------------
    %	Date
    %------------------------------------------------

    \vfill\vfill\vfill % Position the date 3/4 down the remaining page

    {\large\today} % Date, change the \today to a set date if you want to be precise

    \vfill % Push the date up 1/4 of the remaining page

\end{titlepage}

\newpage

\end{document}