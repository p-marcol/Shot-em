%%%%%%%%%%%%%%%%%%%%%%%%%%%%%%%%%%%%%%%%%
% Academic Title Page
% LaTeX Template
% Version 2.0 (17/7/17)
%
% This template was downloaded from:
% http://www.LaTeXTemplates.com
%
% Original author:
% WikiBooks (LaTeX - Title Creation) with modifications by:
% Vel (vel@latextemplates.com)
%
% License:
% CC BY-NC-SA 3.0 (http://creativecommons.org/licenses/by-nc-sa/3.0/)
% 
% Instructions for using this template:
% This title page is capable of being compiled as is. This is not useful for 
% including it in another document. To do this, you have two options: 
%
% 1) Copy/paste everything between \begin{document} and \end{document} 
% starting at \begin{titlepage} and paste this into another LaTeX file where you 
% want your title page.
% OR
% 2) Remove everything outside the \begin{titlepage} and \end{titlepage}, rename
% this file and move it to the same directory as the LaTeX file you wish to add it to. 
% Then add \input{./<new filename>.tex} to your LaTeX file where you want your
% title page.
%
%%%%%%%%%%%%%%%%%%%%%%%%%%%%%%%%%%%%%%%%%

%----------------------------------------------------------------------------------------
%	PACKAGES AND OTHER DOCUMENT CONFIGURATIONS
%----------------------------------------------------------------------------------------
\documentclass[12pt,a4paper,titlepage,hidelinks]{article}

\usepackage[utf8]{inputenc}
\usepackage[polish]{babel}
\usepackage[T1]{fontenc}
\usepackage{subfiles}
\usepackage{float}
\usepackage{hyperref}
\usepackage{color}
\usepackage{amsmath}
\usepackage{hyperref}
\usepackage{graphicx}

\usepackage{mathpazo} % Use the Palatino font by default


\newcommand\myemptypage{
	\null
	\thispagestyle{empty}
	% \addtocounter{page}{-1}
	\newpage
}

\newcommand\pageCounterMinusOne{
	\addtocounter{page}{-1}
}

\title{\Huge{Multiplatformowe Aplikacje Mobilne}}

\author{Jakub Barylak\\Piotr Marcol}
\date{\today}


\begin{document}

\selectlanguage{polish}


\begin{titlepage} % Suppresses displaying the page number on the title page and the subsequent page counts as page 1
    \newcommand{\HRule}{\rule{\linewidth}{0.5mm}} % Defines a new command for horizontal lines, change thickness here

    \center % Centre everything on the page

    %------------------------------------------------
    %	Headings
    %------------------------------------------------

    \textsc{\LARGE Politechnika Śląska}\\[0.5cm] % Main heading such as the name of your university/college

    \textsc{\Large Wydział Automatyki, Elektroniki i Informatyki}\\[1cm] % Major heading such as course name

    \textsc{\large Multiplatformowe Aplikacje Mobilne}\\[0.5cm] % Minor heading such as course title

    %------------------------------------------------
    %	Title
    %------------------------------------------------

    \HRule\\[0.4cm]

    { {\LARGE\bfseries Wymagania funkcjonalne i niefunkcjonalne}\\[0.2cm] {\Large Aplikacja do wewnętrznego udostępniania zdjęć w trakcie wydarzeń}}\\[0.2cm] % Title of your document

    \HRule\\[1.5cm]

    %------------------------------------------------
    %	Author(s)
    %------------------------------------------------

    \begin{minipage}{0.5\textwidth}
        \begin{flushleft}
            % \large
            \textsc{\bfseries
                Jakub Barylak
                \\Piotr Marcol}\\
            Informatyka SSI
            \\sem. 6, ISMIP\\
            Rok akademicki 2023/2024
        \end{flushleft}
    \end{minipage}
    % ~
    \begin{minipage}{0.4\textwidth}
        \begin{flushright}
            % \large
            \textit{Prowadzący}\\
            dr inż. \textsc{\bfseries Michał Sawicki}
        \end{flushright}
    \end{minipage}



    %------------------------------------------------
    %	Logo
    %------------------------------------------------

    % \vfill\vfill
    % \includegraphics[width=0.5\textwidth]{polslLogo.jpeg}\\[1cm] % Include a department/university logo - this will require the graphicx package

    %----------------------------------------------------------------------------------------

    %------------------------------------------------
    %	Date
    %------------------------------------------------

    \vfill\vfill\vfill % Position the date 3/4 down the remaining page

    {\large\today} % Date, change the \today to a set date if you want to be precise

    \vfill % Push the date up 1/4 of the remaining page

\end{titlepage}

\newpage

\section{Opis projektu}

Celem projektu jest stworzenie aplikacji mobilnej, która umożliwi użytkownikom udostępnianie zdjęć w trakcie wydarzeń. Aplikacja będzie działała na systemach Android oraz iOS. Po założeniu konta, organizatorzy będą mogli tworzyć wydarzenia, do których będą mogli zapraszać gości. Następnie każdy będzie miał możliwość dodawania zdjęć, które będą umieszczane we wspólnym albumie wydarzenia. Do każdego ze zdjęć będzie można dodawać komentarze oraz oceny. Po zakończeniu wydarzenia, organizator będzie mógł udostępnić album wszystkim gościom. Od gości nie będzie wymagane zakładanie konta celem większej wygody korzystania z aplikacji. Po określionym czasie od zakończenia wydarzenia goście będą tracili możliwość dodawania zdjęć.

\section{Wymagania funkcjonalne}

\begin{itemize}

    \item Zarządzanie użytkownikami:
          \begin{itemize}
              \item Rejestracja i logowanie z wykorzystaniem OAuth lub adresu \linebreak e-mail
          \end{itemize}

    \item Zarządzanie albumami wydarzeń:
          \begin{itemize}
              \item Tworzenie i edycja albumów
              \item Zapraszanie gości do albumów
              \item Zmiana informacji o wydarzeniu
              \item Usunięcie wydarzenia
              \item Edycja okładki albumu (opis, zdjęcie, lokalizacja, data, tło)
          \end{itemize}

    \item Uczestnictwo w wydarzeniach:
          \begin{itemize}
              \item Dołączanie do wydarzeń za pomocą unikalnego kodu
              \item Dodawanie zdjęć do albumu
              \item Wyświetlanie zdjęć z albumu
              \item Dodawanie komentarzy i ocen do zdjęć
              \item Pobieranie albumu po zakończeniu wydarzenia
          \end{itemize}
          \newpage
    \item Uprawnienia:
          \begin{itemize}
              \item Właściciel ma prawo do usuwania zdjęć, komentarzy i użytkowników z albumu
              \item Ustawianie czasu, po którym goście nie mogą dodawać zdjęć
              \item Udostępnianie albumu wszystkim lub określonym gościom
          \end{itemize}
    \item Dodatkowe funkcje:
          \begin{itemize}
              \item Zmiana języka aplikacji
              \item Dostęp do aparatu, galerii zdjęć i lokalizacji
          \end{itemize}
\end{itemize}

\section{Wymagania niefunkcjonalne}

\begin{itemize}
    \item Design responsywny dla różnych rozdzielczości ekranów
    \item Aplikacja musi działać na urządzeniach z systemem Android 10.0 i nowszymi oraz iOS 12.0 i nowszymi
    \item Płynna praca aplikacji na wszystkich obsługiwanych urządzeniach.
    \item Aplikacja będzie monitorowana pod kątem wydajności i stabilności, a wszelkie błędy będą naprawiane w trybie pilnym
    \item Aktualizacje aplikacji będą dostarczane regularnie w celu poprawy bezpieczeństwa i dostępności
    \item Aplikacja będzie poddawana testom jednostkowym, integracyjnym w celu zapewnienia wysokiej jakości i bezbłędnej pracy
    \item Dane użytkownika będą szyfrowane i przechowywane w bazie danych w sposób bezpieczny
    \item Przy każdym zdjęciu będzie przechowywana informacja o autorze oraz dacie dodania
    \item Aplikacja będzie zgodna z polityką prywatności Google oraz aktualnymi przepisami prawnymi
\end{itemize}

% Kompatybilność:

% Aplikacja musi działać na urządzeniach mobilnych z Androidem 6.0 i nowszymi oraz iOS 12.0 i nowszymi.
% Design responsywny dla różnych rozdzielczości ekranów.

% Wydajność i stabilność:

% Płynna praca aplikacji na wszystkich obsługiwanych urządzeniach.
% Monitorowanie wydajności i stabilności.
% Szybkie usuwanie błędów.

% Aktualizacje:

% Regularne aktualizacje bezpieczeństwa i dostępności.
% Nowe funkcje i ulepszenia na bieżąco.

% Testowanie:

% Szerokie testy jednostkowe i integracyjne.
% Zapewnienie wysokiej jakości i bezbłędnej pracy.

% Bezpieczeństwo:

% Szyfrowanie danych użytkowników.
% Bezpieczne przechowywanie danych w bazie danych.
% Zgodność z polityką prywatności Google i aktualnymi przepisami prawnymi.


\end{document}